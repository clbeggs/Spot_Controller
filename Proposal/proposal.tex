%!TEX TS-program = xelatex

\documentclass{article}

\usepackage{listings}
\usepackage{multicol}
\usepackage{tikz}
\usepackage{fancyhdr} % Fancy headers and footers
\usepackage{amsthm, amsmath, amsfonts, amssymb, mathrsfs, mathtools} % Math packages
\usepackage{enumitem}
\usepackage{geometry}
\usepackage{marginnote}

%% For inkspace figures:
\usepackage{import}
\usepackage{xifthen}
\usepackage{pdfpages}
\usepackage{transparent}
\newcommand{\incfig}[1]{%
    \def\svgwidth{\columnwidth}
    \import{./figures/}{#1.pdf_tex}
}
% \pdfsuppresswarningpagegroup=1

%% Set document margin:
\geometry{margin=1in} %% Add paperheight=16383pt to make page continuous


\newtheorem{theorem}{Theorem}[section]
\newtheorem{corollary}{Corollary}[theorem]
\newtheorem{lemma}[theorem]{Lemma}
\newtheorem*{remark}{Remark}
\theoremstyle{definition}
\newtheorem{definition}{Definition}[section]

\newtheoremstyle{note}{3pt}{3pt}{\normalfont}{}{\bfseries}{:}{ }{}
\theoremstyle{note}
\newtheorem*{note}{Note}

%% Margin Notes
\let\oldmarginpar\marginpar
\renewcommand{\marginpar}[2][text width=3cm, rectangle, draw,rounded corners, thick]{%
        \oldmarginpar{%
        \tikz\node at (0,0) [#1]{#2};}
        }


%% Header and Footer
\pagestyle{fancy}
\fancyhf{}
\lhead{\classname} % Left Header text
\rhead{} % Right header text
\rfoot{Page \thepage}

%\usepackage{fontspec}
%\usepackage{setspace}
%
%\setmainfont{IBMPlexSans}[
%    Path = fonts/,
%    Extension =.ttf,
%    UprightFont = *-Light,
%    BoldFont = *-Bold,
%    ItalicFont = *-Italic,
%    BoldItalicFont = *-BoldItalic
%]


%%% ADD OPTIONAL HEADER AND FOOTER RULES

%%% CHANGE SNIPPETS _ AND ^ TO DETECT IF
%%% {} HAVE ALREADY BEEN WRITTEN


%%% INPUTS %%%%%%%%%%%%%%%%%%%%%%%%%%%%%%%%%%%
\newcommand{\classname}{Open Project Proposal}


%%% SNIPPET COMMANDS %%%%%%%%%%%%%%%%%%%%%%%%%%%%%%%%%%%%%


%%%%%%%%%%%%%%%%%%%%%%%%%%%%%%%% Random stuff
% ital - \textit{}
% bld - \textbf{}
% margin - \marginnote{text}[offset]

%%%%%%%%%%%%%%%%%%%%%%%%%%%%%%%% Random math stuff
% dx  - \frac{\\partial $1}{\\partial $2}
% rarrow - \\rightarrow
% func - \\$1 : \\mathbb{$2} \\rightarrow \\mathbb{$3}, x \\mapsto 
% txt - \text{$1}
% // - \\frac{$1}{$2}$0
% sum -  \\sum \\limits $0
% qed - qed symbol (filled)
% inv - inverse ^{-1}
% math - $ input $ 

%%%%%%%%%%%%%%%%%%%%%%%%%%%%%%%% Environments
% align - \begin{alignedat}{4}
% props - \align{} - Numbered
% bm - \\begin{bmatrix} $1 \\end{bmatrix}
% pm - \\begin{pmatrix} $1 \\end{pmatrix}
% thm - \\begin{thm}{$1} $1 \end{thm}
% def - \\begin{def}{$1} $1 \end{def}
% lemma - \begin{lemma}
% cor - corollary \begin{corollary}
% proof - \begin{proof}
% remark - \begin{remark}
% dm -  \[\]
% beg - \begin{$1} \end{$1}
% python - \begin{lstlisting}[language=Python,escapeinside={(*}{*)}, basicstyle=\fontsize{11}{13}]
% section -  \section{}
% subsection -  \subsection{}

%%%%%%%%%%%%%%%%%%%%%%%%%%%%%%%% Constants
% reals - \mathbb{R}
% rationals - \mathbb{Q}
% complex - \mathbb{C}
% eps -  \\epsilon
% sig - \\sigma
% Sig - \\Sigma
% prime - ^{\prime}
% alpha - \\alpha
% beta - \\beta



%%%% START OF DOCUMENT ############
\begin{document}


\section*{Goals:}
Use state of the art gait generation and motion planning to control Boston Dynamics Spot in Webots to 
walk on uneven terrain while being robust to various external pushes/forces. 
The evaluation of the system will be having Spot walk on unseen and technical terrain to reach a goal state, 
with added pushes at random times.



\subsection*{Roadmap}
\begin{enumerate}[label=\arabic*.), start=1] 
\item  Set up Webots sim in flat environment, use perfect sensors and actuators, and utilize MoveIt
to have Spot walk around.
\item Replace MoveIt with SOTA gait generation methods - (Alternate Method: Train multiple policies with RL, have high level controller to choose policy)
\item Add noisy actuators and sensors to more resemble reality
\item Replace flat world with uneven terrain
\item Make gait generation methods robust to external forces (First on flat world)
\end{enumerate}

\subsection*{Teammates}
\begin{itemize}
    \item Chris Beggs - Took UMich's DL course, (Slowly) going through Sergey Levine's Deep RL course, plus I'm going through literature about this stuff to prepare.
\end{itemize}

\subsection*{Stretch Goals/Things I could add}
\begin{itemize}
\item Add obstacles to world and use SLAM
\item Compare gait generation methods with Deep RL
\item Various challenges on final project competition
\item Socially Aware Motion Planning
\item Add manipulator for object grasping
\end{itemize}


\subsection*{Motivation}
I do want to work on a robot with joints/manipulators because it's an interesting problem.
Robots like Spot are better equiped for technical terriain that cars are not able to reach. Also, 
smaller quadrepeds are able to achieve a better relationship with humans than cars.
Plus I love dogs.

\subsection*{Timeline Risks}
Mostly just some aspect of the project taking much longer than anticipated, like implementing some aspects
of different papers, unexpected software issues, etc.



\subsection*{Related Work}
\begin{itemize}
\item \textbf{Gait Generation:}
    \begin{itemize}
    \item Discovery of Complex Behaviors through Contact-Invariant Optimization \textit{Mordatch et al.}
    \item Feature-Based Locomotion Controllers \textit{Martin de Lasa et al.}
    \item Fast and Flexible Multilegged Locomotion Using Learned Centroidal Dynamics (Source code included) \textit{Kwon et al.} - http://calab.hanyang.ac.kr/papers/flexLoco.html
    \item Gait and Trajectory Optimization by Self-Learning for Quadrupedal Robots with an Active Back Joint \textit{Masuri et al.}
    \item Automatic  Gait  Pattern  Selection for  Legged  Robots, \textit{Wang et al.}
    \item Adaptation  of  Quadruped  Gaits  Using  Surface Classification  and  Gait  Optimization \textit{Kim et al.}
    \item A Robust Quadruped Walking Gait for  Traversing Rough Terrain \textit{Pongas et al.}
    \item Robust  Gait  Synthesis  Combining  Constrained  Optimization  and Imitation Learning \textit{Ding et al.}
    \item Gait and Trajectory Optimization for Legged Systems Through Phase-Based End-Effector Parameterization \textit{Winkler et al.}
    \end{itemize}
\item \textbf{SLAM} 
    \begin{itemize}
        \item ORB-SLAM3: An Accurate Open-Source Library for Visual, Visual-Inertial and Multi-Map SLAM
        \item ORB-SLAM2: an Open-Source SLAM System for Monocular, Stereo and RGB-D Cameras
    \end{itemize}
\end{itemize}




\end{document}




