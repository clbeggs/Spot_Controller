\documentclass{article}

\usepackage{listings}
\usepackage{multicol}
\usepackage{tikz}
\usepackage{fancyhdr} % Fancy headers and footers
\usepackage{amsthm, amsmath, amsfonts, amssymb, mathrsfs, mathtools} % Math packages
\usepackage{enumitem}
\usepackage{geometry}
\usepackage{marginnote}

%% For inkspace figures:
\usepackage{import}
\usepackage{xifthen}
\usepackage{pdfpages}
\usepackage{transparent}
\newcommand{\incfig}[1]{%
    \def\svgwidth{\columnwidth}
    \import{./figures/}{#1.pdf_tex}
}
\pdfsuppresswarningpagegroup=1

%% Set document margin:
\geometry{margin=1in} %% Add paperheight=16383pt to make page continuous


\newtheorem{theorem}{Theorem}[section]
\newtheorem{corollary}{Corollary}[theorem]
\newtheorem{lemma}[theorem]{Lemma}
\newtheorem*{remark}{Remark}
\theoremstyle{definition}
\newtheorem{definition}{Definition}[section]

\newtheoremstyle{note}{3pt}{3pt}{\normalfont}{}{\bfseries}{:}{ }{}
\theoremstyle{note}
\newtheorem*{note}{Note}

%% Margin Notes
\let\oldmarginpar\marginpar
\renewcommand{\marginpar}[2][text width=3cm, rectangle, draw,rounded corners, thick]{%
        \oldmarginpar{%
        \tikz \node at (0,0) [#1]{#2};}%
        }


%% Header and Footer
\pagestyle{fancy}
\fancyhf{}
\lhead{\classname} % Left Header text
\rhead{} % Right header text
\rfoot{Page \thepage}


%%% ADD OPTIONAL HEADER AND FOOTER RULES

%%% CHANGE SNIPPETS _ AND ^ TO DETECT IF
%%% {} HAVE ALREADY BEEN WRITTEN


%%% INPUTS %%%%%%%%%%%%%%%%%%%%%%%%%%%%%%%%%%%
\newcommand{\classname}{Adv. Robotics Writeup}


%%% SNIPPET COMMANDS %%%%%%%%%%%%%%%%%%%%%%%%%%%%%%%%%%%%%


%%%%%%%%%%%%%%%%%%%%%%%%%%%%%%%% Random stuff
% ital - \textit{}
% bld - \textbf{}
% margin - \marginnote{text}[offset]

%%%%%%%%%%%%%%%%%%%%%%%%%%%%%%%% Random math stuff
% dx  - \frac{\\partial $1}{\\partial $2}
% rarrow - \\rightarrow
% func - \\$1 : \\mathbb{$2} \\rightarrow \\mathbb{$3}, x \\mapsto 
% txt - \text{$1}
% // - \\frac{$1}{$2}$0
% sum -  \\sum \\limits $0
% qed - qed symbol (filled)
% inv - inverse ^{-1}
% mmath - $ input $ 

%%%%%%%%%%%%%%%%%%%%%%%%%%%%%%%% Environments
% align - \begin{alignedat}{4}
% props - \align{} - Numbered
% bm - \\begin{bmatrix} $1 \\end{bmatrix}
% pm - \\begin{pmatrix} $1 \\end{pmatrix}
% thm - \\begin{thm}{$1} $1 \end{thm}
% def - \\begin{def}{$1} $1 \end{def}
% lemma - \begin{lemma}
% cor - corollary \begin{corollary}
% proof - \begin{proof}
% remark - \begin{remark}
% dm -  \[\]
% beg - \begin{$1} \end{$1}
% python - \begin{lstlisting}[language=Python,escapeinside={(*}{*)}, basicstyle=\fontsize{11}{13}]
% section -  \section{}
% subsection -  \subsection{}

%%%%%%%%%%%%%%%%%%%%%%%%%%%%%%%% Constants
% reals - \mathbb{R}
% rationals - \mathbb{Q}
% integers - \mathbb{Z}
% complex - \mathbb{C}
% eps -  \\epsilon
% sig - \\sigma
% Sig - \\Sigma
% prime - ^{\prime}
% alpha - \\alpha
% beta - \\beta



%%%% START OF DOCUMENT ############
\begin{document}


\section{Abstract}
This project tackled the problem of quadruped locomotion and navigation.
There are many different ways to solve this problem, control and trajectory
optimization, deep reinforcment learning, etc. 
But in an attempt to maintain generality, Model Agnostic Meta Learning
proposed by Finn et. al\cite{finn} was used. 

\section{Introduction}

\section{Related Work}
\cite{winkler} utilized rigid body dynamics. 


MAML 
MAML++

In policy gradient optimization, \cite{trpo} 

TRPO
VPG
PPO




\section{Methodology}
MAML was replicated close to what the original paper.
Some stability tricks were used from \cite{maml++}, 
such as the increased gradient steps. 
Ómaml


\subsection{Model Architecture}
\textbf{Policy Net}\\
Fully connected network, 2 hidden layers with sizes 256, ReLU 
activation functions.\\\\
\textbf{Value Net}\\
Fully connected network, 2 hidden layers with sizes 128, ReLU 
activation functions.\\\\
\subsection{Training}
As training was done on a laptop, networks had to be kept at the smallest possible size. 200 epochs, 1 task, and 5 sampled trajectories took 2 hours to train.



\section{Results}
\section{Discussion}
\section{Conclusion}
\section{Refernces}













\end{document}




